\documentclass{article}
\usepackage[utf8]{inputenc}

\usepackage{sectsty}
\usepackage{fancyhdr}
\usepackage[top=1.25cm, bottom=2cm, left=2cm, right=2cm]{geometry}

\title{Cahier des charges}
\begin{document}

%permet de souligner un certain type de partie
\sectionfont{\underline}

\maketitle
\section{Présentation du Projet}
	\subsection{Contexte et définition du problème}
		Exposer le problème et le scénario pour lequel on développe cette application.\\
		dans notre cas c'est une demande d'un client qui veut réaliser des rondes avec un drone et avoir un retour vidéo en direct.
	\subsection{Objectifs}
		Lister les objectifs à atteindre en les chiffrant précisément.\\
		On veut par exemple une latence vidéo entre le drone et l'ipod de l'ordre de la milliseconde pas plus.
\section{Expression des besoins}
	\subsection{Besoins fonctionnels}
		Fonctions que le logiciel doit réaliser.\\
		Tableau des fonctionnalités listées par ordre d'importance.\\
		Surtout pas de code dans ce tableau juste une définition précise et succincte de ce que réalisera la fonction.
	\subsection{Besoins non fonctionnels}
		Autre besoin qui n'exprime pas une fonction du logiciel.\\
		Cela peut être par exemple la contrainte de performance qui dans notre cas est le fait d'avoir un retour vidéo fluide et HD.\\
		Exemple : besoin d'une interface 
\section{Contraintes}
	\subsection{Matériel}
		Moyen matériels et logiciel mis a disposition.\\
		On ne parle pas de budget car il n'y en a pas c'est un projet universitaire.\\
	\subsection{Délais}
		Date de livraison du produit.\\
		Autres Échéances du projet par exemple le rendu du cahier des charges.
		Contrainte de temps : contrainte la plus importante.\\
		(Lister les contraintes par ordre d'importance)
	\subsection{Autres contraintes}
		Autres contraintes a prendre en compte s'il y en a sinon on supprime cette partie.


\end{document}
