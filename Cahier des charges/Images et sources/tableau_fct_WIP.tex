\documentclass{article}
\usepackage[utf8]{inputenc}

\usepackage{tabularx}
\usepackage{fancyhdr}
\usepackage[top=2cm, bottom=2cm, left=2cm, right=2cm]{geometry}

\date{}
\title{Tableau des fonctionnalités complet}

\begin{document}
\maketitle

\renewcommand{\arraystretch}{1.75}
\begin{center}
\begin{tabularx}{\linewidth}{|c|X|}
	\hline
	1 & Saisie intuitive du plan de vol par exemple à la souris.\\
	\hline
	2 & Cliquer à un endroit sur la carte permet d'ajouter un point de passage (waypoint).\\
	\hline
	3 & Les points de passages sont ordonnés selon leur ordre de création.\\
	\hline
	4 & L'altitude de chaque point de passage doit être modifiable.\\
	\hline
	5 & Chaque point de passage peut être déplacé par exemple via un glissé déposé à la souris.\\
	\hline
	6 & Un waypoint doit pouvoir être supprimé.\\
	\hline
	7 & Le plan de vol peut être supprimé en entier.\\
	\hline
	8 & Le plan de vol saisi doit être récupérable sous la forme d'un fichier.\\
	\hline
	9 & La navigation sur la carte doit comprendre différents niveau de zoom pour permettre de saisir des points géographiques précis.\\
	\hline
	10 & Chaque plan de vol peut être sauvegardé pour une utilisation futur.\\
	\hline
	11 & Une interface doit permettre d'accéder aux plans de vol sauvegardés.\\
	\hline
	12 & L'exécution du plan de vol doit pourvoir être lancée via une action simple par exemple un bouton ou un geste particulier fait avec le masque FPV.\\
	\hline
	13 & L'exécution d'un plan de vol doit pouvoir être arrêtée en cas d'urgence.\\
	\hline
	14 & Le retour vidéo de la caméra du drone doit être visualisé sur un Ipod au travers d'un masque FPV.\\
	\hline
	15 & Le retour vidéo doit être en haute définition et doit avoir une latence minime de l'ordre de la seconde.\\
	\hline
\end{tabularx}
\end{center}

\end{document}