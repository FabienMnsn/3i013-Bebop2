\documentclass{article}
\usepackage[utf8]{inputenc}

\usepackage{tabularx}
\usepackage{fancyhdr}
\usepackage[top=1cm, bottom=1cm, left=1cm, right=1cm]{geometry}

\date{}
\title{Tableau des use case VS fonctionnalités}

\begin{document}
\maketitle
\begin{center}
\renewcommand{\arraystretch}{1.75}
\begin{tabularx}{16cm}{|X|X|}
	\hline
	Use Case & Fonctionnalité\\
	\hline
	L'utilisateur à besoin de définir un plan de vol précis & Création d'une application Linux permettant la saisie d'un plan de vol\\
	\hline
	L'utilisateur veut dessiner graphiquement le plan de vol sur une carte & Possibilité d'ajouter et de supprimer des waypoints sur une carte\\
	\hline
	L'utilisateur veut conserver son plan de vol & Système de sauvegarde et de gestion des plan de vols\\
	\hline
	L'utilisateur valide la saisie de son plan de vol & Communiquer le plan de vol au drone\\
	\hline
	L'utilisateur veut exécuter son plan de vol & Système de gestion des plans de vol présents sur le drone\\
	\hline
	L'utilisateur veut observer le flux vidéo (capturé par le drone) en direct & Transmission du flux vidéo vers l'Ipod\\
	\hline
	L'utilisateur veut arrêter l'exécution du plan de vol & Fonction d'arrêt d'urgence facile d'accès\\
	\hline
\end{tabularx}

\end{center}
\end{document}