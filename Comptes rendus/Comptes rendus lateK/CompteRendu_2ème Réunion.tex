\documentclass{article}
\usepackage[utf8]{inputenc}

\usepackage{fancyhdr}
\usepackage[top=3cm, bottom=3cm, left=3cm, right=3cm]{geometry}

\title{Compte-rendu de réunion}
\date{8 Février 2019}

\begin{document}

\maketitle

\section{Concernant les exposés}
\subsection{Contenu de la présentation}
\bigbreak
Tout d'abord, il est essentiel que les membres du groupe se présentent au début de l'exposé.
\bigbreak
Le sujet doit être introduit et présenté clairement au début de la présentation, un plan est bienvenu si l'exposé est suffisamment long. Lequel cas, il faut prévoir une slide du plan ordonné correctement.
Il est important de bien organiser l'exposé pour éviter au maximum les redites.
\bigbreak
Il faut présenter les besoins et contrainte du client, ainsi que les cas d'utilisation du produit ("use cases"). Le projet et ses objectifs doivent être intélligibles pour des personnes externes au projet. Il faut donc faire attention à l'usage de vocabulaire et de jargon propre à l'équipe, ainsi qu'aux dérives de langage.
\bigbreak
Lors de l'étude d'une piste de développement, si elle n'est pas demandé par le client, s'assurer de sa pertinence. Ce qui n'est pas "facturable" n'est pas à aborder.
\bigbreak

\subsection{Organisation}
Le fil conducteur de la présentation est le cahier des charges. Il faut construire l'exposé autour et toujours être en lien avec ce dernier.
\bigbreak
L'ordre des sujets traités est important. Il faut structurer l'exposé, numéroter les sujets et les relier au fil conducteur, afin que ce dernier soit clair et facile à suivre pour le public.
\bigbreak
Quand il y a des schémas, il ne faut pas dupliquer les slides. Un seul schéma clair et complet, dont les éléments apparaissent au fur-et-à-mesure des explications est préférable.
\bigbreak
\section{Concernant le projet}
\bigbreak
Il est important de faire une note pour chaque outil utilisé, détaillant l'usage fait de ce dernier par l'équipe. Tout éventuel nouveau membre du projet doit avoir à sa disposition les documentations suffisantes pour s'adapter à l'usage des outils collaboratifs de l'équipe.
\bigbreak 
Certaines questions sur des détails précis, comme les outils pour réaliser l'interface du logiciel, les outils pour le transfert vidéo vers l'Ipod etc, ne sont pas forcément nécessaires dans un premier temps.
\bigbreak
Le cahier des charges doit être rédigé pour début Mars.

\end{document}
