\documentclass[]{article}

\usepackage{fancyhdr}
\usepackage[top=3cm, bottom=3cm, left=3cm, right=3cm]{geometry}

\begin{document}

\pagestyle{fancy}
\renewcommand\headrulewidth{1pt}
\fancyhead[L]{3I013 - Projet }
\fancyhead[R]{Sorbonne Université}
\renewcommand\footrulewidth{1pt}
\fancyfoot[L]{Licence Informatique}
\fancyfoot[R]{Vendredi 8 Mars 2019}

\begin{center}
\large{\textbf{Compte Rendu Réunion \\ Vendredi 8 Mars 2019 \bigbreak}}
\end{center}

\section{Concernant l'organisation de la présentation}
En ce qui concerne le comparatif des architectures étudiées, le choix et l'argumentation de la solution choisie son corrects. La présentation est de meilleure qualité et plus organisée que les fois précédentes, néanmoins quelques points sont tout de même à corriger.\\
\indent Il faut éviter de mentionner tous les fragments du use-case dans le plan, étant donné qu'il s'agit d'un seul et même use-case, bien qu'il soit étalé sur plusieurs diapositives.\\
Il manque une story entre la présentation du projet et le use case.\\
\indent Nous ne nous posons pas assez de questions du point de vue ergonomique, typiquement l'arrêt d'urgence par le clavier n'est pas optimisée.\\
\\
\indent Il faut exposer des chiffres sur lesquels nous pourrons nous appuyer lorsque nous parlons par exemple, de latence vidéo.\\
\indent Pour finir, il ne faut pas recommencer une nouvelle présentation à chaque réunion comme si ce que l'on avait fait précédemment n'avait pas existé. Il faut que l'on puisse observer les évolutions par rapport à la dernière réunion. \\
\\
\indent Lorsqu’un problème est identifié et qu'il n'a pas été spécifié par le client, nous pouvons proposer une solution.\\


\section{Concernant le projet}
En ce qui concerne l'arrêt d'urgence, l'activation par le clavier est à éviter car peu ergonomique sauf si un second opérateur est présent pour assister l'utilisateur portant le casque. Mais dans ce cas il faut le préciser dans le cahier des charges.\\
Il est préférable de nous diriger vers un mouvement spécifique de la tête.\\
\\
\indent Le prototype de l'application de saisie du plan de vol est correct.\\
\indent Les salles 409 et 508 du master SAR des tours 14/15 sont à notre disposition pour travailler de 9h à 20h30.\\


\section{Concernant le cahier des charges}
Dans le cahier charges, il faut numéroter et légender les images.\\
\indent On pourra incruster le tableau du comparatif des solutions dans le cahier des charges, ainsi qu'un texte détaillant les arguments mis en avant.\\

\end{document}