\documentclass[]{article}

\usepackage{fancyhdr}
\usepackage[top=3cm, bottom=3cm, left=3cm, right=3cm]{geometry}

\begin{document}

\pagestyle{fancy}
\renewcommand\headrulewidth{1pt}
\fancyhead[L]{3I013 - Projet }
\fancyhead[R]{Sorbonne Université}
\renewcommand\footrulewidth{1pt}
\fancyfoot[L]{Licence Informatique}
\fancyfoot[R]{Vendredi 8 Mars 2019}

\begin{center}
\large{\textbf{Compte Rendu Réunion \\ Vendredi 8 Mars 2019 \bigbreak}}
\end{center}
	

La présentation est de meilleure qualité et plus organisée que les fois précédentes, néanmoins quelques points sont tout de même à corriger.\\
Il manque une story entre la présentation du projet et le use case.\\
Nous ne nous posons pas assez de questions du point de vue ergonomique, typiquement l'arrêt d'urgence par le clavier n'est pas optimisée.\\
\\

Il faut exposer des chiffres sur lesquels nous pourrons nous appuyer lorsque nous parlons par exemple, de latence vidéo.\\
Également, nous devons préciser si l'utilisateur est seul ou accompagné dans le use case.\\
Pour finir, il faut que l'on puisse observer les évolutions quant à la dernière réunion. \\
\\
\\
Lorsqu’un problème est identifié et qu'il n'a pas été spécifié par le client, nous pouvons proposer une solution.\\
En ce qui concerne l'arrêt d'urgence, l’invocation par le clavier est à proscrire, nous devons nous diriger sur des mouvements spécifiques de la tête. \\
Nous pourrons aussi demander à un deuxième utilisateur d'activer ce mode, à la condition de le préciser dans le cahier des charges.\\
\\
\\
Les salles 409 et 508 du master SAR des tours 14/15 sont à notre disposition pour travailler de 9h à 20h30.


\end{document}