\documentclass[]{article}

\usepackage[utf8]{inputenc}
\usepackage{fancyhdr}
\usepackage[top=3cm, bottom=3cm, left=3cm, right=3cm]{geometry}

\begin{document}

\pagestyle{fancy}
\renewcommand\headrulewidth{1pt}
\fancyhead[L]{3I013 - Projet }
\fancyhead[R]{Sorbonne Université}
\renewcommand\footrulewidth{1pt}
\fancyfoot[L]{Licence Informatique}
\fancyfoot[R]{Samedi 13 Avril 2019}

\begin{center}
\large{\textbf{Compte Rendu de la Réunion du Vendredi 12 Avril 2019 \bigbreak}}
\end{center}

\section{Concernant la présentation}
On passe trop de temps sur le plan. Il faut réduire le plan au principaux points de la présentation.\\

La présentation du contexte du projet et de la demande du client est "triste". Elle pourrai être beaucoup plus professionnelle, et bien plus vendeur.\\

Le use case ne présente qu'un seul cas, celui avec une seule personne utilisant  l'application. On ne présente pas d'autres cas d'utilisation par exemple avec deux personne.\\
Présenter toutes les solutions d'architectures étudiées n'est pas très pertinent à ce stade du projet. Il vaut mieux présenter l'architecture adoptée et aborder rapidement les autres architectures.\\

La présentation de l'architecture logicielle et mieux que celle de la réunion précédente. les parties sont affichées une à une pour arriver, à la fin au schéma complet de l'architecture logicielle.\\

Lors de l'affichage des différentes parties du schéma d'architecture il faut éviter que les images bouges entre deux diapos.\\

Le schéma d'avancement du projet et trop flou. Que veut dire "en cours de développement"?. Il vaut mieux opter pour un genre de graphique avec les pourcentages d'avancement estimés.

La slide de présentation des problèmes rencontrés est correcte. Il manque un schéma pour bien comprendre le problème rencontré avec OpenCV.\\

Pas de conclusion pourquoi?\\

On n'aborde toujours pas le déploiement de l'application.
Comment allons-nous faire?
On pourrai générer une archive avec la machine de développement et ce serai la machine du client qui installerai l'archive.\\

La prochaine réunion se tiendra le vendredi 19 Avril 2018.

\end{document}
