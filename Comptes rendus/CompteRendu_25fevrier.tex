\documentclass{article}
\usepackage[utf8]{inputenc}

\title{Compte-rendu de réunion}

\date{25 février}
\begin{document}
\maketitle


\section{Concernant les exposés}
\subsection{Forme de la présentaion}
\bigbreak
Abordons en premier lieu la forme des slides de présentations. Il faudrait éviter d'utiliser Beamer pour réaliser ces derniers ou alors le customiser afin d'enlever les élements inutiles qui prennent beaucoup de place. Il ne faut pas passer trop de temps sur le sommaire et les diapos doivent absoluement être numérotées pour des raisons pratiques.
\bigbreak
La présentation n'est pas assez professionnelle, nous devons nous mettre en situation d'une véritable proposition de produit à un client afin de nous préparer au monde du travail et de la recherche.
\bigbreak

\subsection{Contenu de la présentation}
\bigbreak
Concernant le contenu de la présentation, il manque toujours un 'use case' complet, ce qui correspond à un sénario concret d'utilisation, cela permettra d'établir le cahier des charges. L'exposé doit ainsi être structuré autour de ce 'use case'.

\bigbreak
Il faut faire attention à ne pas se dévaloriser dans la proposition des différentes solutions et ne pas proposer une solution contenant autant d'avantages que d'inconvénients.
\bigbreak
La synthaxe de l'exposé doit être uniforme, les conventions établits doivent être respectées. Pour finir il ne faut pas oublier de détails importants.

\bigbreak
