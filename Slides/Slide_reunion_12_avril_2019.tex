%________________________________________________________________________________________________
\documentclass{beamer}
%________________________________________________________________________________________________
\usepackage[french]{babel} 
\usepackage[utf8]{inputenc} 
\usepackage[T1]{fontenc} 
\usepackage{graphicx}
\usepackage[utf8]{inputenc}
\usepackage{fancyhdr}
\usepackage{geometry}
\usepackage{tabularx,tabulary}
\usepackage{movie15}
%________________________________________________________________________________________________
\title{3I013 Réunion du 12 Avril 2019}
\author{Nicolas CASTANET\\Maël FRANCESCHETTI\\Daoud KADOCH\\Fabien MANSON\\}
%________________________________________________________________________________________________
%ce theme est le plus clean de Beamer le truc a ne pas utiliser c'est 'Warsaw'
\usetheme{default}
%suppression de la barre de navigation inutile
\setbeamertemplate{navigation symbols}{}
\setbeamertemplate{frametitle}[default][center]

%\logo{\includegraphics[height=0.5cm]{logo_sorbonne.png}}

%________________________________________________________________________________________________
\addtobeamertemplate{footline}{
	\begin{flushright}
	\vbox{\insertframenumber/\inserttotalframenumber}
	\end{flushright}}

%________________________________________________________________________________________________
\begin{document}


	\begin{frame}
		\begin{center}
		\date{}
		\maketitle
		\end{center}
	\end{frame}
	
%________________________________________________________________________________________________	
	
	\begin{frame}
		\section{}
		\begin{center}
		\frametitle{Sommaire}
		\tableofcontents{}
		\end{center}
	\end{frame}
	
	%________________________________________________________________________________________________
	
	\begin{frame}
		\section{La Demande du Client}
		\begin{center}
		\frametitle{La Demande du Client}
		\begin{itemize}
		    \item Le client souhaite effectuer des rondes avec un drone Bebop 2\\
		    \item Le drone doit voler de manière autonome en suivant un plan de vol prédéfini \\
		    \item Le retour vidéo du drone doit être redirigé à un iPod touch qui sera placé dans un masque FPV pour permettre à l'utilisateur de voir comme s'il était à la place du drone\\
		\end{itemize}
		   
		\end{center}
	\end{frame}
%________________________________________________________________________________________________	
	\begin{frame}
		\section{Scénario d'Utilisation}
		\begin{center}
		\frametitle{Scénario d'utilisation}
		\begin{enumerate}
		    \item Démarrage du drone\\
		    \item Lancement de l'application PC\\
		    \item Saisie du plan de vol à l'aide du composant dédié (carte interactive)\\
		    \item Connexion du PC au wifi du drone
		    \item Connexion de l'iPod au réseau local et démarrage de l'application sur l'iPod
		    \item Mise en place de l'iPod dans le masque FPV
		    \item Démarrage de la ronde depuis l'iPod ou le PC
		    \item Arrêt d'urgence si besoin
		\end{enumerate}
		   
		\end{center}
	\end{frame}
	
%________________________________________________________________________________________________	

	\begin{frame}
		\section{Les Différentes Solutions}
		\begin{center}
		\frametitle{Comparatif des solutions}
		%\subsection{Contraintes}
        %\framesubtitle{Les solutions}
       
        \includegraphics[scale=0.9]{comparatif_v3.PNG}
        \begin{itemize}
            \item Solution retenue : PC + iPod\\
        \end{itemize}
		\end{center}
	\end{frame}
	
%________________________________________________________________________________________________

	\begin{frame}
		\section{Architecture Matérielle}
		\begin{center}
		\frametitle{Architecture Matérielle}
		%\subsection{Contraintes}
        %\framesubtitle{Les solutions}
       
        \includegraphics[scale=0.6]{schema_archi.png}
		\end{center}
	\end{frame}
	
%________________________________________________________________________________________________	

	\begin{frame}
		\section{Architecture Logicielle}
		\begin{center}
		\frametitle{Architecture Logicielle}
		%\subsection{Contraintes}
        %\framesubtitle{Les solutions}
       
        \includegraphics[scale=0.3]{Architecture_logicielle_v2.jpg}
		\end{center}
	\end{frame}
	
%________________________________________________________________________________________________
	
	
	\begin{frame}
		\section{Test Effectués}
		\begin{center}
		\frametitle{Tests effectués}
           	\begin{itemize}
           	    \item Saisie d'un plan de vol enregistré au format Mavlink
                \item Connexion au drone
                \item Envoi d'un fichier Mavlink avec le SDK
                \item Initialisation des paramètres pour le vol autonome
                \item Exécution du plan de vol enregistré sur le drone en totalité.
                 \item Arrêt d'urgence
                \item Essais d'émission d'un flux vidéo depuis un serveur vers un iPod
                \item Traitement d'un flux vidéo en direct (webcam) -> découpe et assemblage des images pour le format VR 
            \end{itemize}
		\end{center}
	\end{frame}

%________________________________________________________________________________________________

	\begin{frame}
		\section{État d'Avancement}
		\begin{center}
		\frametitle{Etat d'avancement}
         \includegraphics[scale=0.35]{Avancement_projet.PNG}
        \end{center}
	\end{frame}
	
%________________________________________________________________________________________________	
	
	\begin{frame}
		%\section{Avancement}
		\begin{center}
		\frametitle{Problèmes rencontrés}
	    \begin{itemize}
                 \item Résolus :
            \end{itemize}
        \begin{enumerate}
             
            \item Calibration du drone après chaque arrêt d'urgence (choc ou inclinaison trop forte du drone)
             \item Direction du drone durant le trajet
             \end{enumerate}
            \begin{itemize}
                \item En cours :
            \end{itemize}
            \begin{enumerate}
           \item Perte du signal GPS sur campus de l'UPMC
           \item Encodage du fichier video traité en .avi : conversion en mp4 (ffmpeg) trop lente
           \item Ré-émission du flux vidéo par le serveur
           \end{enumerate}
		\end{center}
	\end{frame}

%________________________________________________________________________________________________	
	
\end{document}
