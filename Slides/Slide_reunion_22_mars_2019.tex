%________________________________________________________________________________________________
\documentclass{beamer}
%________________________________________________________________________________________________
\usepackage[french]{babel} 
\usepackage[utf8]{inputenc} 
\usepackage[T1]{fontenc} 
\usepackage{graphicx}
\usepackage[utf8]{inputenc}
\usepackage{fancyhdr}
\usepackage{geometry}
\usepackage{tabularx,tabulary}
\usepackage{movie15}
%________________________________________________________________________________________________
\title{3I013 Réunion du 22 Mars 2019}
\author{Daoud KADOCH\\Fabien MANSON\\Maël FRANCESCHETTI\\Nicolas CASTANET\\}
%________________________________________________________________________________________________
%ce theme est le plus clean de Beamer le truc a ne pas utiliser c'est 'Warsaw'
\usetheme{default}
%suppression de la barre de navigation inutile
\setbeamertemplate{navigation symbols}{}
\setbeamertemplate{frametitle}[default][center]

%\logo{\includegraphics[height=0.5cm]{logo_sorbonne.png}}

%________________________________________________________________________________________________
\addtobeamertemplate{footline}{
	\begin{flushright}
	\vbox{\insertframenumber/\inserttotalframenumber}
	\end{flushright}}

%________________________________________________________________________________________________
\begin{document}


	%premiere diapo
	\begin{frame}
		\begin{center}
		\date{21 mars 2019}
		\maketitle
		Ceci est la première diapo !\\
		\end{center}
	\end{frame}
	
	
	
	\begin{frame}
		\section{}
		\begin{flushleft}
		\frametitle{Sommaire}
		\tableofcontents{}
		\end{flushleft}
	\end{frame}
	
	
	\begin{frame}
	\section{Mavlink}
		\begin{center}
		\frametitle{Mavlink}
		Ceci est la troisieme diapo !\\
		\end{center}
	\end{frame}
	
	\begin{frame}
	\section{iPod et video en local}
		\begin{center}
		\frametitle{Stream reseau local sur iPod}
		Incruster ici les avancées sur la lecture de vidéo sur le réseau local depuis l'iPod.
		\end{center}	
	\end{frame}
	
	\begin{frame}
	\section{Prototype interface graphique}
		\begin{center}
		\frametitle{Prototype d'interface graphique Gtk+}		
		\includegraphics[scale=0.35]{schema_GUI.png}
		\end{center}
	\end{frame}
	
\end{document}